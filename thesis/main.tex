\documentclass[twoside, 12pt]{article}
\usepackage{xdipp}
\english
\pismo{Latin Modern}
\begin{document}
\titul{MicroPython Utilizing Zephyr Port and NXP FRDM-MCXN947}{Dmitrii Titarenko}{Ing. Jan Kolomazník, Ph.D.}{Brno 2025}

\podekovani{I would like to express my gratitude to everyone who contributed to completion of 
this thesis.\\ I would like to thank my thesis supervisor, Ing. Jan Kolomazník, Ph.D. , for 
his guidance and support. Furthermore, I express my gratitude to Ph.D. Zbynek Fedra for 
guiding me during the creation of this thesis.\\
I am also very grateful to all the professors in the Department of Informatics at Mendel 
University for their teaching, openness, and constant support throughout my study.\\
A big thank you to family members and friends for their support and encouragement.\\
Finally, I express gratitude to the people who directly or indirectly contributed to the 
creation of this work. To those who came before me and on whose shoulders I stand.}
\prohlasenimuz{Brno ~03.08.2025}

\abstract{Titarenko, D. MicroPython Utilizing Zephyr Port and NXP FRDM-MCXN947. Bachelor 
thesis. Brno, 2025.}{This thesis explores MicroPython support on Zephyr RTOS using the NXP 
FRDM-MCXN947 development board. Zephyr RTOS itself provides extended support and easy-to-use 
APIs to many embedded devices and their peripherals, but the support of MicroPython remains 
limited and not consistent with MicroPython ports developed for target devices. This work 
analyzes the current limitations of MicroPython on Zephyr RTOS and selects specific 
functionality for implementation. The implementation involves initializing Zephyr and 
MicroPython development environments adjusted to FRDM-MCXN947, conducting functional testing 
of the development board peripherals with Zephyr and MicroPython environment, and comparing 
the comparability of native and Zephyr ports of MicroPython.}
\keywords{MicroPython, thesis, Zephyr RTOS, FRDM-MCXN947}

\obsah
\kapitola{Introduction}
The world of microcontrollers and embedded devices continues to grow, and such devices become 
more common with every day. Even though they are seldom noticeable we more often might find 
ourselves surrounded by them. From home appliances, to cars, to factory machines to city-wide 
networks embedded devices reach wide and deep in our lives.

But with the growing count of embedded devices grows complexity of functions they implement. 
Hence arises a need for Operating Systems(OSs) to manage sets of complex programs and provide 
a layer of abstraction to ease development in such constrain but demanding environments.

To cover demand of an Operating System in embedded devices the Zephyr Real Time Operating 
System (RTOS) was created. Zephyr RTOS is an open-source operating system with build-in 
security and optimization for resource limited devices. Zephyr kernel supports ARM, Intel 
x86, ARC, RISC-V, Nios II, Tensilica Xtensa and large number of development boards, among 
others NXPs FRDM-MCXN947. ALso Zephyr has rich API that allows developers to write high-level 
code for embedded devices.

Writing software for embedded devices is still a complex task regardless of what underlying 
technologies are used. Writing it in a language such as C adding an additional complexity
due to need of managing program memory allocation and deallocation by hand.
Leaving unhandled memory sector could lead to memory leaks or worse opens an opportunity for
an attacker to execute malicious code on an embedded device. The consequences of such problems
become much grater when occurring in the embedded world. MicroPython is an optimized subset 
of Python 3 programming language for embedded devices. It aims to ease writing software by 
managing memory using its garbage collector system, using easy to read Python-like syntax and 
providing various modules to enable work with different peripherals. Additionally, 
MicroPython allows for code portability, meaning that code written for FRDM-MCXN947
could be ported and ran on ESP32 with minimal updates to code. 

But MicroPython does not 
support every single device straightaway -- ported versions of MicroPython are submitted to 
the MicroPython repository, by a manufacturer or an enthusiasts. Later submitted port will be 
reviewed and tested by MicroPython maintainers, which is a lengthy process, for example 
MicroPython port for Zephyr was under review for 2 years.

By combining Zephyr RTOS and MicroPython in one technological stack the best of both 
technologies could be utilized. Potential exists for developers to write highly readable,
easy-to-understand and efficient code with MicroPython that make use of various hardware 
support introduced by Zephyr RTOS. Yet the state of this development environment is not yet 
firm and have plenty of rough edges and unapparent problems that could arise during the 
process of software development.

The aim of this thesis is to construct a method for configuring the development environment 
for MicroPython, Zephyr RTOS and FRDM-MCXN947 development board, provide insight of 
compatibility challenges of both platforms and propose potential solutions for extending 
MicroPython port to Zephyr RTOS. The finding of this thesis will contribute to understanding 
of the state of both platforms and their integration, informing future developers and increasing usability of MicroPython and Zephyr.

\kapitola{Background}
This chapter introduces the reader to an information about Zephyr RTOS, MicroPython and FRDM-
MCXN947 board that is needed for understanding this thesis.
\sekce{Zephyr RTOS}
Zephyr is an Operation System designed for resource-constrained and embedded 
system from simple sensors to smart industrial embedded solutions. It supports a broad list 
of embedded devices, development boards and peripherals. Zephyr offers extensive number of 
features and services including multi-threading, inter-thread data passing, inter-thread 
synchronization, dynamic memory allocation, interrupt service, power management, networking.
Zephyr project is open-source, distributed under Apache 2.0 license and was created under 
Linux Foundation organization.\cite{zephyr1}

\begin{literatura}
\citace{zephyr1}{Zephyr Project -- Intorduction}{Zephyr Project -- Intorduction [onlne] Zephyr Project, 2025 Available from: https://docs.zephyrproject.org/latest/introduction/index.html}
\end{literatura}
\end{document}