\documentclass[twoside, 12pt]{article}
\usepackage{xdipp}
\usepackage{listings}
\english
\pismo{Latin Modern}


\lstdefinestyle{mystyle}{
    numbers=left,                    
    numbersep=5pt,
}
\lstset{style=mystyle}

\begin{document}
\titul{MicroPython Utilizing Zephyr Port and NXP FRDM-MCXN947}{Dmitrii Titarenko}{Ing. Jan Kolomazník, Ph.D.}{Brno 2025}

\podekovani{I would like to express my gratitude to everyone who contributed to completion of 
this thesis.\\ I would like to thank my thesis supervisor, Ing. Jan Kolomazník, Ph.D. , for 
his guidance and support. Furthermore, I express my gratitude to Zbynek Fedra Ph.D. for 
guiding me during the creation of this thesis.\\
I am also very grateful to all the professors in the Department of Informatics at Mendel 
University for their teaching, openness, and constant support throughout my study.\\
A big thank you to family members and friends for their support and encouragement.\\
Finally, I express gratitude to the people who directly or indirectly contributed to the 
creation of this work. To those who came before me and on whose shoulders I stand.}
\prohlasenimuz{Brno ~03.08.2025}

\abstract{Titarenko, D. MicroPython Utilizing Zephyr Port and NXP FRDM-MCXN947. Bachelor 
thesis. Brno, 2025.}{This thesis explores MicroPython support on Zephyr RTOS using the NXP 
FRDM-MCXN947 development board. Zephyr RTOS itself provides extended support and easy-to-use 
APIs to many embedded devices and their peripherals, but the support of MicroPython remains 
limited and not consistent with MicroPython ports developed for target devices. This work 
analyzes the current limitations of MicroPython on Zephyr RTOS and selects specific 
functionality for implementation. The implementation involves initializing Zephyr and 
MicroPython development environments adjusted to FRDM-MCXN947, conducting functional testing 
of the development board peripherals with Zephyr and MicroPython environment, and comparing 
the comparability of native and Zephyr ports of MicroPython.}
\keywords{MicroPython, thesis, Zephyr RTOS, FRDM-MCXN947}

\obsah

\listoffigures

\kapitola{Introduction}
The world of microcontrollers and embedded devices continues to grow, and such devices become 
more common with every day. Even though they are seldom noticeable we more often might find 
ourselves surrounded by them. From home appliances, to cars, to factory machines to city-wide 
networks embedded devices reach wide and deep in our lives.

But with the growing count of embedded devices grows complexity of functions they implement. 
Hence arises a need for Operating Systems(OSs) to manage sets of complex programs and provide 
a layer of abstraction to ease development in such constrain but demanding environments.

To cover demand of an Operating System in embedded devices the Zephyr Real Time Operating 
System (RTOS) was created. Zephyr RTOS is an open-source operating system with build-in 
security and optimization for resource limited devices. Zephyr kernel supports ARM, Intel 
x86, ARC, RISC-V, Nios II, Tensilica Xtensa and large number of development boards, among 
others NXPs FRDM-MCXN947. ALso Zephyr has rich API that allows developers to write high-level 
code for embedded devices.

Writing software for embedded devices is still a complex task regardless of what underlying 
technologies are used. Writing it in a language such as C adding an additional complexity
due to need of managing program memory allocation and deallocation by hand.
Leaving unhandled memory sector could lead to memory leaks or worse opens an opportunity for
an attacker to execute malicious code on an embedded device. The consequences of such problems
become much grater when occurring in the embedded world. MicroPython is an optimized subset 
of Python 3 programming language for embedded devices. It aims to ease writing software by 
managing memory using its garbage collector system, using easy to read Python-like syntax and 
providing various modules to enable work with different peripherals. Additionally, 
MicroPython allows for code portability, meaning that code written for FRDM-MCXN947
could be ported and ran on ESP32 with minimal updates to code. 

But MicroPython does not 
support every single device straightaway -- ported versions of MicroPython are submitted to 
the MicroPython repository, by a manufacturer or an enthusiasts. Later submitted port will be 
reviewed and tested by MicroPython maintainers, which is a lengthy process, for example 
MicroPython port for Zephyr was under review for 2 years.

By combining Zephyr RTOS and MicroPython in one technological stack the best of both 
technologies could be utilized. Potential exists for developers to write highly readable,
easy-to-understand and efficient code with MicroPython that make use of various hardware 
support introduced by Zephyr RTOS. Yet the state of this development environment is not yet 
firm and have plenty of rough edges and unapparent problems that could arise during the 
process of software development.

The aim of this thesis is to construct a method for configuring the development environment 
for MicroPython, Zephyr RTOS and FRDM-MCXN947 development board, provide insight of 
compatibility challenges of both platforms and propose potential solutions for extending 
MicroPython port to Zephyr RTOS. The finding of this thesis will contribute to understanding 
of the state of both platforms and their integration, informing future developers and increasing usability of MicroPython and Zephyr.

\sekce{Goal of this thesis}
Goal of the thesis is to establish development environment and workflow for developing 
applications for embedded devices with the MicroPython Zephyr port. This work could be used
in future to ease start of application development and as a reference.

\kapitola{Background}
This chapter introduces the reader to an information about Zephyr RTOS, MicroPython and FRDM-
MCXN947 board that is needed for understanding this thesis.

\sekce{Zephyr RTOS}
Zephyr is an Operation System designed for resource-constrained and embedded 
system from simple sensors to smart industrial embedded solutions with emphasis on safety. It 
supports a broad list 
of embedded devices, development boards and peripherals. Zephyr offers extensive number of 
features and services including multi-threading, inter-thread data passing, inter-thread 
synchronization, dynamic memory allocation, interrupt service, power management, networking, file system.
Zephyr project is open-source, distributed under Apache 2.0 license and was created under 
Linux Foundation organization.\cite{zephyr1}

\obrazek
\vlozobr{images/security-zephyr-system-architecture}{0.5}
\endobr{Zephyr System Architecture \obrzdroj{\cite{zephyr_architecture}}}

\podsekce{West}
West is a part of Zephyr's tool-chain used for building and configuring. West can initiate
Zephyr workspace from official upstream repository, update or change version of a local
Zephyr workspace to any version in official repository, build Zephyr application from source,
flash built application to a board.\cite{zephyr_west}

\podsekce{Kconfig}
Kconfig is Zephyr's kernel, peripheral drivers and subsystems configuration system that allow 
to configure Zephyr at a build time. Kconfig goal is to enable configuration without 
introducing changes to the source
code.

The initial board configuration can be found in \textbf{<board>\_defconfig} files. For
example configuration file for FRDM-MCXN947 is located at \textbf{boards/nxp/frdm\_mcxn947/
frdm\_mcxn947\_mcxn947\_cpu0\_defconfig}. The board configuration for NXPs' FRDM-MCXN947
is as follows:
\begin{lstlisting}{caption=FRDM-MCXN947 Kconfig configuration}
CONFIG_CONSOLE=y
CONFIG_UART_CONSOLE=y
CONFIG_SERIAL=y
CONFIG_UART_INTERRUPT_DRIVEN=y
CONFIG_GPIO=y
CONFIG_PINCTRL=y
CONFIG_ARM_MPU=y
CONFIG_HW_STACK_PROTECTION=y
CONFIG_TRUSTED_EXECUTION_SECURE=y
\end{lstlisting}

Kconfig values can be set to a \textbf{<board>\_defconfig} files, temporarily with terminal
graphical interfaces or with a \textbf{prj.conf} file at application level which overrides
the initial configuration during application build.\cite{zephyr_kconfig}

\podsekce{Devicetree}
Devicetree is a data structure to describe hardware. It is a community driven standard
that is heavily used in Zephyr project. In Zephyr devicetrees are usually build inherently 
meaning that for example FRDM-MCXN947 has a devicetree configuration \textbf{board/nxp/frdm\_mcxn947/frdm\_mcxn947\_mcxn947\_cpu0.dts}
which mainly enables peripheral devices, but includes FRDM-MCXN947 specific configuration 
from \textbf{frdm\_mcxn947.dtsi} (include file), which in turn includes 
\textbf{frdm\_mcxn947-pinctrl.dtsi} file that mostly defines pinmux groups. Additionally the 
\textbf{frdm\_mcxn947\_mcxn947\_cpu0.dts} includes \textbf{nxp\_mcxn94x.dtsi} file that 
defines memory ranges for SRAM, FLEXSPI and peripherals and includes 
\textbf{nxp\_mcxn94x\_common.dtsi} include file where most of devices including CPU, GPIO, 
CTIMER and others are defined and assigned memory ranges.\cite{devicetree_spec}

Same as Kconfig Devicetrees can be overwritten or have some specific devices configured 
differently with \textbf{overlay} files, which as well needs to be placed in build directory, 
from there \textbf{west tool} will use it to edit the Devicetree configuration.

\sekce{MicroPython}
MicroPython is an open-source project founded by Damien George. MicroPython is an implementation of the Python programming 
language that 
is optimized to be run on embedded and resource constraint devices. It implements 
the entire Python 3.4 syntax with some selected features from the later versions such as
\textbf{async/await} from Python 3.5 , additionally on par with Python it uses garbage 
collection system for memory management. MicroPython final build include a compiler that 
compiles MicroPython code to bytecode and an runtime interpreter of the compiled bytecode.
Programs could be written directly to the MicroPython REPL(Read–eval–print loop) or be loaded
onto MicroPython host device with use of serial connection and utility programs like \textbf{ampy}.


MicroPython's core development is focused on implementing and maintaining core features
of the MicroPython like Python language features, libraries, memory management and 
MicroPython interpreter. The responsibility for adapting and porting MicroPython to different
platforms lies on the community around it. Every MicroPython port introduces required 
adaptations and addresses hardware features and limitations of its platform. Consequently,
MicroPython support is not linear on all platforms, because some might lack the necessary
configuration for enabling a part of functionality or even lack reimplementation of a number of core libraries. Additionally, the slow pace of adding to source code features for various platforms 
created by the community means that
even fully functional and tested ports or features might wait for months before being reviewed.
Despite all of this, there are already many supported devices and architectures that 
MicroPython can run on. MicroPyton has additional support to be run on
operating system Zephyr RTOS and on OSes from UNIX family, as well as experimental Windows 
port.

MicroPython remains in beta-stage, hence it is a subject to possible API and code-base changes in the future.\cite{mpy_book}

\sekce{FRDM-MCXN947}
The FRDM-MCXN947 is a low-cost development board designed by NXP semiconductors. FRDM-MCXN947 
integrates Dual Arm
Cortex-M33 microcontroller, a neural processing unit, P3T1755DP I3C temperature sensor, 
TJA1057GTK/3Z CAN PHY, Ethernet PHY, SDHC circuit, RGB LED, touch pad, high-speed USB, MCU-Link debuger,
push buttons and has an option to be extended with external devices.
\cite{mcxn947_manual}

\obrazek
\vlozobr{images/frdm-diagram}{0.35}
\endobr{FRDM-MCXN947 Block diagram \obrzdroj{\cite{mcxn947_web}}}

\podsekce{Signal Multiplexing}
FRDM-MXCN947 enables use of several functions for different pins by utilizing Signal 
Multiplexing. For example pin \textbf{P0\_10} which is an red RGB pin can use \textbf{GPIO} 
functionality directly, \textbf{FLEXCOMM} by utilizing \textbf{FC0\_P6} FLEXCOMM device,
\textbf{CTIMER} by utilizing \textbf{CT0\_MAT0} CTIMER device, and \textbf{FLEXIO} 
functionality by utilizing \textbf{FLEXIO0\_D2} device.

Only one function can be used at a time on a pin and only one pin can be assigned to a 
peripheral device. \cite{mcx_manual} 

\podsekce{LinkServer}
LinkServer is an NXP command-line utility that provides  target flashing capabilities and 
firmware updates for FRDM-MCXN947. \cite{link_server}

\sekce{MicroPython port to Zephyr}
For a long time the MicroPythons' Zephyr port have been using an older versions of Zephyr and
MicroPython itself. It used MicroPython 1.19.1 and Zephyr 3.1.0 versions which both came out 
in period between May and June 2022 until September 2024. In September 2022 began work by 
Maureen Helm  to introduce a CI pipeline into MicroPython repository to ease porting 
MicroPython to latest Zephyr release. From this work emerged last MicroPython Zephyr port 
version based on MicroPython 1.24.0 and Zephyr 3.7.0.

And though MicroPython Zephyr port
already supports the MicroPython modules like socket, time, math, machine and other are 
implemented and usable in the final MicroPython build they may lack support of some 
sub-modules like machine's PWM sub-module or functionality of modules and sub-modules like
not yet implemented features of machine's I2C sub-module that do not have ability to set clock and data lines. 


\begin{literatura}
\citace{zephyr1}{Zephyr Project, Intorduction, 2024}{Zephyr Project, Intorduction [onlne], 2024 Available from: https://docs.zephyrproject.org/latest/introduction/index.html}

\citace{zephyr_west}{Zephyr Project, West (Zephyr’s meta-tool), 2024}{Zephyr Project, West (Zephyr’s meta-tool) [online], 2024 Available from: https://docs.zephyrproject.org/latest/develop/west/index.html}

\citace{zephyr_kconfig}{Zephyr Project, Configuration System (Kconfig), 2022}{Zephyr Project, Configuration System (Kconfig) [onlne], 2022 Available from: https://docs.zephyrproject.org/latest/build/kconfig/index.html}

\citace{mcx_manual}{NXP semiconductors, MCX Nx4x Reference Manual, 2025}{NXP semiconductors, MCX Nx4x Reference Manual, 2025 MCXNX4XRM}

\citace{devicetree_spec}{devicetree.org, Devicetree Specification Release v0.4, 2023}{devicetree.org, Devicetree Specification Release v0.4, 2023}

\citace{mpy_book}{Nicholas H. Tollervey, 2017}{\autor{Nicholas H. Tollervey} \nazev{Programming with MicroPython: embedded programming with microcontrollers and Python}. O'Reilly Media, Inc., 2017. ISBN 978-1-491-97273-1}

\citace{mcxn947_manual}{NXP semiconductors, UM12018 FRDM-MCXN947 Board User Manual, 2024}{NXP semiconductors, UM12018 FRDM-MCXN947 Board User Manual, 2024}

\citace{link_server}{NXP semiconductors -- LinkServer for Microcontrollers, 2025}{NXP semiconductors, LinkServer for Microcontrollers [online], 2025 \\ Available from : https://www.nxp.com/design/design-center/software/development-software/mcuxpresso-software-and-tools-/linkserver-for-microcontrollers:LINKERSERVER}

\citace{mcxn947_web}{NXP semiconductors, FRDM Development Board for MCX N94/N54 MCUs , 2025}{NXP semiconductors, FRDM Development Board for MCX N94/N54 MCUs [online], 2025 Available from: https://www.nxp.com/design/design-center/development-boards-and-designs/FRDM-MCXN947}

\citace{zephyr_architecture}{Zephyr Project, Zephyr Security Overview, 2024}{Zephyr Project, Zephyr Security Overview [onlne], 2024 Available from: https://docs.zephyrproject.org/latest/security/security-overview.html}
\end{literatura}
\end{document}